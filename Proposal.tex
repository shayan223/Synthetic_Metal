

\documentclass[11pt]{article}
\usepackage{acl2015}
\usepackage{times}
\usepackage{url}
\usepackage{latexsym}
\usepackage{scalerel}
\usepackage{graphicx,xparse}

%\setlength\titlebox{5cm}

\title{CS510 NLP - Synth Metal \emojione{}{\scalerel*{\includegraphics{1F918}}{X}}}

\author{Lawrence Gunnell \\
  Portland State University \\
  905 sw Ceder Hills blvd \\
  {\tt lgunnell@pdx.edu} \\\And
  Santiago Tobon \\
  Portland State University \\
  9930 sw 151st ave \\
  {\tt stobon@pdx.edu} \\ \\\And
  Shayan Jalalipour \\
  Portland State University \\
  16766 sw Marcile ln \\
  {\tt shayan2@pdx.edu} \\}
\date{}

\begin{document}
\maketitle


\section{Research Paper Survey}

\subsection{Bibliographical Info}
\subsection{Background}
The research paper entitled "Dopelearning" utilizes deep learning methods to generate substantive and meaningful lyrics for rap-based songs. The rankSVM and deep neural networks were used to create verses possessing both semantic meaning and above-average rhyming density. 
\subsection{Summary of Contributions}
\subsection{Limitations and Discussion}
\subsection{Why This Paper?}
The clarity of the paper in both detailing the algorithmic implementation, as well as in identifying and extracting features, is very accessible and relevant to our project. While not all of the songs we will be training on will maintain strict rhyming schemes, their methods for training the network to detect certain sentence structures or properties of words serves as a guide for engineering our own feature extraction methods.
\subsection{Wider Research Context}


\section{Proposed Project}

\subsection{Main Goal}
The main goal of this project is to create a model that can generate heavy metal song lyrics resembling as closely as possible lyrics written by people.
\subsection{NLP Task} %what NLP task will you address
This project addresses two (albeit similar) tasks: Machine mimicry and "Creativity". These go hand in hand, first the objective is to mimic human made lyrics with enough accuracy and realism as to create genuine lyrics following common lyric and word patterns. Next is creativity. We can apply the model to create novel lyrics, where novelty is its deviation from the initial training set of lyrics, while maintaining as much "Human" mimicry as possible.
\subsection{Data} %What data will you use?
We will use Kaggle's "Large Metal Lyrics Archive" of approximately 228k songs. It is important to note however that not all of these have lyrics, it is simply a compilation of many bands and their respective albums and songs.
\subsection{Methods} %what mathods are we planning to use
Our current plan is to modify and apply GPT-2 for the purpose of better identifying relationships between words and to further supplement training of neural networks to predict varying n-grams of lyrics ahead of an initial word or phrase.
\subsection{Baseline} %baseline methods we will use
Because our goal is to generate text, we can use a simple Naive Bayes N-gram model to generate lyrics based on probability and word frequency as observed throughout the corpus.
\subsection{Evaluation} %How will you evaluate your results
Different ways to measure "Realism" include the model's ability to form rhymes, verses and chorus, as well as maintain context, theme, phrasing, and subjects throughout a given song.


% include your own bib file like this:
%\bibliographystyle{acl}
%\bibliography{acl2015}

\begin{thebibliography}{}

\bibitem[\protect\citename{Aho and Ullman}1972]{Aho:72}
Alfred~V. Aho and Jeffrey~D. Ullman.
\newblock 1972.
\newblock {\em The Theory of Parsing, Translation and Compiling}, volume~1.
\newblock Prentice-{Hall}, Englewood Cliffs, NJ.

\bibitem[\protect\citename{{American Psychological Association}}1983]{APA:83}
{American Psychological Association}.
\newblock 1983.
\newblock {\em Publications Manual}.
\newblock American Psychological Association, Washington, DC.



\end{thebibliography}

\end{document}
